\documentclass[12pt]{article}
\usepackage[utf8]{inputenc}
\usepackage[T1]{fontenc}
\usepackage[french]{babel}


\begin{document}

\title{Rendu 1 Projet Technologique L3}
\author{Théo Dupont}
\maketitle


\bigskip

Voici les fonctions implémentées des 3 premiers TP.\\
Tous les tests sont déroulés sur 3 images différentes toutes implémentées dans l'application, 
qui est elle même testée avec un AVD du Pixel 2 XL, sous la version 8.1(Oreo).
Les temps de test donnés seront sur la photo "Eye", hormis pour l'extension de dynamique en couleur, 
qui sera effectuée sur la photo "Red Gradient".\\
Puisque j'ai réalisé l'étape de profiling sur mon ordinateur personnel (qui n'est pas aussi performant que les appareils du CREMI), il se peut que les temps ne soient pas exacts.\\
\\
\\



L'application peut afficher 3 photos différentes en appuyant sur les 3 boutons: Eye, Mario et Red Gradient respectivement.\\
L'application affichera automatiquement la hauteur et la largeur de l'image.\\
Le bouton reset permet d'afficher l'image de base sélectionnée auparavant.\\
\\
Le bouton blackline affiche une fine bande noir au travers de l'image. Temps: 0.40 seconde\\
Le bouton To Gray applique la première version de To Gray (avec getPixel). Temps: 17.74 secondes\\
Le bouton To Grays applique quand à lui la seconde version de To Gray (avec getPixels). Temps: 0.99 seconde \\
\\
La fonction Colorize(2.1) convertit l'image en HSV pour modifier la teinte aléatoirement, puis repasser l'image en RGB Temps: 1.79 seconde\\
Keep color (2.2) retient aléatoirement 60 degrés du cercle HSV, et passe le reste en gris. La première photo peut être toute grise car elle ne possède pas certaines couleurs. Temps: 2.18 secondes.\\
L'extension de dynamique en gris (3.1.1) augmente le contraste a partir de l'algorithme du cours. Temps: 3.94 secondes\\
puis en couleur qui réalise la même chose sur les 3 canaux RGB(3.1.3),Temps: 4.23 secondes\\
L'égalisation d'histograme en gris (3.2.1) aplanit l'histograme avec un autre algorithme vu en cours. Temps: 2.62 secondes\\
puis en couleur (3.2.2). Temps: 4.30 secondes.\\
\\
Toutes ces fonctions sont bien implémentées et fonctionnent correctement.
\\
Cependant, la fonction de diminution de contraste ne fonctionne que partiellement. (3.1.2)\\
La fonction Hsv To Color possède également un problème de conversion sur la Hue, mais elle marche parfaitement pour Colorize car elle ne nécessite pas cette information.\\



\end{document}